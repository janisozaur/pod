\documentclass{classrep}
\usepackage[utf8]{inputenc}
\usepackage[pdftex]{graphicx}
\usepackage[polish]{babel}
\usepackage{algorithm}
\usepackage{algorithmic}
\usepackage{multicol}
\usepackage{amsmath}
\usepackage{listings}
\usepackage{array}
\usepackage{multirow}
\usepackage{url}
\usepackage{subfigure}
\usepackage{hyperref}

\studycycle{Informatyka, studia dzienne, II st.}
\coursesemester{I}

\coursename{Przetwarzanie obrazów i dźwięków}
\courseyear{2010/2011}

\courseteacher{dr inż. Bartłomiej Stasiak}
\coursegroup{micmic}
\svnurl{https://serce.ics.p.lodz.pl/svn/labs/poid/bsat_wt1415/micmic}

\author{%
  \studentinfo{Michał Janiszewski}{169485} \and
  \studentinfo{Michał Kawski}{169487}
}

\title{Zadanie 2: Filtracja w dziedzinie częstotliwości}

\floatname{algorithm}{Algorytm}

\begin{document}

\maketitle

\section{Cel zadania}
Celem zadania było rozwinięcie aplikacji powstałej w wyniku wykonania zadania 1. o możliwości filtrowania obrazów w dziedzinie częstotliwości.

Wariant nam przydzielony przewidywał implementację następujących transformat:
\begin{itemize}
 \item proste i odwrotne szybkie przekształcenie Fouriera,
 \item proste i odwrotne (odpowiednio rodzaje 2 i 3) szybkie przekształcenie kosinusowe.
\end{itemize}

Ponadto, należało stworzyć zestaw filtrów umożliwiający dokonywanie operacji na przekształconych obrazach.

\section{Opis metod przetwarzania}
\subsection{Przekształcenie Fouriera}
Przekształceniem Fouriera sygnału ciągłego, jednowymiarowego, $f(x)$ nazywamy $F(u)$ zdefiniowane w następujący sposób:
\begin{equation}
  F(u) = \int_{-\infty}^{\infty} f(x)e^{-j2\pi ux}dx
\end{equation}

Odwrotne przekształcenie Fouriera dla tak powstałego sygnału $F(u)$ zdefiniowane jest w następujący sposób:
\begin{equation}
  f(x) = \int_{-\infty}^{\infty}F(u) e^{j2\pi ux}du
\end{equation}

Dla dyskretnego sygnału wejściowego $\{f_0, f_1, \ldots, f_{N - 1}\}$, powyższa prosta transformata przyjmuje następującą postać:
\begin{equation}
  \label{eq.dft}
  F(n) = \displaystyle \sum_{k = 0}^{N - 1} f(k) e^{-i\frac{2\pi}{N}kn}
\end{equation}
gdzie $N$ określa ilość elementów sygnału wejściowego.

Transformata odwrotna dla sygnału dyskretnego określona jest wzorem:
\begin{equation}
  \label{eq.idft}
  f(k) =\frac{1}{N} \displaystyle \sum_{n = 0}^{N - 1} F(k) e^{i\frac{2\pi}{N}kn}
\end{equation}

Dla dwumiarowego sygnału dyskretnego $L(n, m)$, takiego jak np. obraz cyfrowy, transformata przyjmuje postać:
\begin{equation}
  \label{eq.dft2d}
  F(j, k) = \displaystyle\sum_{n = 0}^{N - 1} \displaystyle \sum_{m = 0}^{M - 1} L(n, m) e^{-i\frac{2\pi}{N} nj} e^{-i\frac{2\pi}{M}mk}
\end{equation}

Podobnie odwrotna transformata określona jest wzorem:
\begin{equation}
  \label{eq.idft2d}
  L(n, m) = \frac{1}{MN} \displaystyle \sum_{j = 0}^{N - 1} \displaystyle \sum_{k = 0}^{M - 1} F(j, k) e^{i\frac{2\pi}{N} nj} e^{i\frac{2\pi}{M}mk}
\end{equation}
gdzie $M$ i $N$ to rozmiary sygnału.

Ponieważ transformata Fouriera jest separowalna, wzór \ref{eq.dft2d} można rozpisać jako transformatę transformat:
\begin{equation}
  F(j, k) = \displaystyle \sum_{n = 0}^{N - 1} \left[\displaystyle \sum_{m = 0}^{M - 1} L(n, m) e^{-i\frac{2\pi}{M}mk} \right] e^{-i\frac{2\pi}{N} nj}  
\end{equation}
zaś wzór \ref{eq.idft2d} jako:
\begin{equation}
  L(n, m) = \frac{1}{MN} \displaystyle \sum_{j = 0}^{N - 1} \left[\displaystyle \sum_{k = 0}^{M - 1} F(j, k) e^{i\frac{2\pi}{M}mk} \right] e^{i\frac{2\pi}{N} nj}
\end{equation}

\subsubsection{Szybkie przekształcenie Fouriera}
Zgodnie ze wzorem \ref{eq.dft} łatwo zauważyć, że złożoność algorytmu dla sygnału jedno wymiarowego wynosi $O(N^2)$\footnote{Notacja wielkiego $O$.}, zaś dla dwuwymiarowego (wzór \ref{eq.dft2d}) $O(N^3)$.

Istnieją jednak efektywniejsze algorytmy implementujące DFT (\emph{DFT - discrete Fourier transform}) nazywane ,,szybkim przekształceniem Fouriera'' (\emph{FFT - fast Fourier transform}). Opierają się one o fakt, że transformatę można rozpisać w następujący sposób:

\begin{equation}
  \label{wzor.fft}
  \begin{array}{c}
    F(n) = \displaystyle \sum_{k=0}^{N-1} f(k) e^{-i\frac{2\pi}{N}nk} = \displaystyle\sum_{k = 0}^{N/2 - 1} f(2k) e^{-i\frac{2\pi}{N}(2k)n} + \displaystyle\sum_{k = 0}^{N/2 - 1} f(2k + 1) e^{-i\frac{2\pi}{N}(2k + 1)n} \\
    = \displaystyle\sum_{k = 0}^{N/2 - 1} f(2k) e^{-i\frac{2\pi}{N/2}kn} + e^{-i\frac{2\pi}{N}n}\displaystyle\sum_{k = 0}^{N/2 - 1} f(2k + 1) e^{-i\frac{2\pi}{N/2}kn}
  \end{array}
\end{equation}

W wyniku takiego rozpisania otrzymujemy dwie transformaty o długości $N / 2$ każda. Pierwsza zawiera parzyste elementy ciągu, zaś druga \ppauza nieparzyste. Rozbicie takie pozwala zredukować złożoność obliczeniową algorytmu do $O(N\log_2 N)$.

\subsection{Dyskretne przekształcenie kosinusowe}
Dyskretne przekształcenie kosinusowe dla sygnału dyskretnego $f(x)$ zdefiniowane jest następująco:
\begin{equation}
  \label{eq.dct}
  C(u) = \alpha (u) \displaystyle \sum_{x = 0}^{N - 1} f(x) \cos \left(\frac{\pi(2x + 1)u}{2N}\right)
\end{equation}
dla $u = 0, 1, 2, \ldots, N - 1$, natomiast $\alpha(u)$ oznacza:
\begin{equation}
  \label{eq.dctalpha}
  \alpha(u) = \begin{cases} \sqrt{\frac{1}{N}} & \text{$u = 0$} \\
			    \sqrt{\frac{2}{N}} & \text{$u \neq 0$}
              \end{cases}
\end{equation}

Przekształcenie odwrotne do \ref{eq.dct} opisane jest wzorem:
\begin{equation}
  \label{eq.idct}
  f(x) = \displaystyle \sum_{u = 0}^{N - 1} \alpha(u) C(u) \cos \left(\frac{\pi(2x + 1)u}{2N}\right)
\end{equation}
dla $x = 0, 1, 2, \ldots, N - 1$.

\clearpage


\end{document}